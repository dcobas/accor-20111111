\documentclass[a4paper]{article}
\usepackage{graphicx,paralist,algorithmic,times,mathptmx}
\usepackage{algorithm,paralist}
\usepackage{amsmath}
% \usepackage[margin=1in]{geometry}

\parskip=3pt
\parindent=0pt
\newcommand\fname[1]{\texttt{#1}}

\title{A note on PCI addressing in the CCDB}
\author{Juan David Gonz\'alez Cobas}

\begin{document}
\maketitle

\section{Purpose and scope}

The intention of this document is to give a clear explanation of the
problem of (physically) describing PCI configurations in the CCDB.
The current solution for VME modules is shortly described in
section~\ref{vme}. Section~\ref{pci} describes the problems that PCI
poses to physical addressing. The partial solutions to the problem
current in place are explained in section~\ref{crateconfigs}.
The last section is devoted to a proposal on how to integrate these in
the CCDB.


\section{VME Crate Description in the CCDB}\
\label{vme}

The configuration of VME crates in the CCDB is input via an Oracle
form. The crucial data provided to identify a module sitting in a crate
are
\begin{compactitem}
\item VME slot
\item module type
\item logical unit number (LUN)
\end{compactitem}
From this, and the declaration of the module type, we obtain the basic
parameters needed for a driver to address the module:
\begin{compactitem}
\item base address(es)
\item interrupt vector(s)
\end{compactitem}
These parameters are usually set up via jumpers in the boards, as
described pictorially in images linked from the CCDB.

All this information is extracted into a \fname{transfer.ref} file
describing the boot sequence of the machine. Drivers are provided the
necessary information at load (boot) time by their install programs;
usually, these install programs take this information from
\fname{transfer.ref}.

Note that the information in the database is inserted or used at
different moments
\begin{compactitem}
\item hardware setup time: boards are set up via jumpers, settings
described in pictorial form
\item crate configuration time: boards are compactenumd in an Oracle form,
their relevant parameters described therein
\item boot time: driver installation programs use \fname{transfer.ref}
to extract the information relevant to them and pass it into the driver.
\end{compactitem}

From the driver point of view, the really crucial information to take
into account is the mapping from a hardware module type and a LUN, to a
set of module parameters like the base address:
\begin{equation*}
\text{HWTYPE}\quad\text{LUN} \to \text{ADDRESS}
\end{equation*}
Drivers and application software will correctly identify and address
modules as long as the driver gets this mapping right.

Note that in this correspondence, the physical slot (number from 1 to
21 in a typical VME crate) is irrelevant, and is usually misconfigured
without breaking this correspondence. However, this datum is crucial to
locate the modules when setting up the crate, configuring it, or
performing interventions. The use case to keep in mind is a crate with
10 identical modules (differing only in their jumper settings) that we
need to identify by LUN. It is crucially important that the
correspondence between the physical slot in the crate and the LUN gets
through the driver.

\section{PCI Peripherals Geographical Description}
\label{pci}

A growing need exists for a geographical description of PCI devices
analogous to the one shown in section~\ref{vme}.

Unfortunately, addresses and module-specific configuration are
established at boot time by the BIOS and (sometimes) the operating
system. The practical consequence is that the closest we get to a
physical index for a module in the PCI tree is a descriptor by bus,
device and function number. We will refer to this triad as BDF from now
on; suffice to say that, once the computer has booted, every PCI device
is uniquely identified by it in a BIOS-dependent manner.

The problem here is: two industrial PCs identical except for the CPU or
even the BIOS could assign different BDF triads to the same physical
slot in the crate.

So, the BDT triad plays the same role as the physical slot in VME
crates, but with two downsides
\begin{compactitem}
\item switching CPUs or BIOSes would change the mapping 
\begin{equation*}
\text{PHYSICAL SLOT} \leftrightarrow \text{BDF}
\end{equation*}
\item it is not immediately obvious what BDF corresponds to a physical
slot, unless we go about a tedious procedure (like examining the 
output of utilities like \fname{lspci} with different board setups)
\end{compactitem}
Given that the only way for a driver to tell one device from the other
among a set of identical modules is the BDF, we are facing two problems:
first, a CPU or BIOS change would potentially break the assumed topology
of a system. Second, a system integrator must have some way to number
devices that ensures the correspondence between their LUNs and their
physical position in the crate.

\section{Some fixes}

\label{crateconfigs}


\end{document}

