\documentclass[compress,red]{beamer}
\usetheme{Warsaw}
\useoutertheme[subsection=false]{smoothbars}

\usepackage{helvet}
\usepackage{graphicx}
\usepackage{booktabs}
\usepackage{hyperref}
\usepackage{url}

\usepackage{remreset}% tiny package containing just the \@removefromreset command
\makeatletter
\@removefromreset{subsection}{section}
\makeatother
\setcounter{subsection}{1}

\title{Drivers and Database Aspects}
\author{Juan David Gonz\'alez Cobas\\
	BE/CO Device Drivers Team}

\begin{document}

\begin{frame}
\titlepage
\end{frame}

\section*{Outline}
\begin{frame}
\tableofcontents
\end{frame}

\section{Discontinuation of XML-based driver load}

\begin{frame}{Driver installation process}
In a typical VME driver scenario, a driver needs some information from
user space to locate devices and initialize.

\pause
The configuration description for the crate is in the CCDB and is
extracted into a \url{transfer.ref} file by a program called
\url{dscinit}

\pause
This way, a device driver knows the parameters (base addresses,
interrupt vectors, etc.) for each of the devices it has to manage
\end{frame}

\begin{frame}{The file \url{drivers.xml}}

A new mechanism for passing this information was proposed
(beginning 2009?) to completely describe the hardware layout of a crate

\pause
It turned out to have many disadvantages

\pause
\begin{itemize}
\item inconsistencies between \url{transfer.ref} and \url{drivers.xml}
\item incomplete representation of information
\item need to embed an XML parser in the driver load process
\item (mis)implementation of a complex install
\item introduction of superfluous dependencies
\item \url{nmn} was right
\end{itemize}

\pause
As a consequence, the use of the \url{drivers.xml} mechanism is
deprecated and will be migrated to a \url{transfer.ref}-based
mechanism by Q1 2012.
\end{frame}

\begin{frame}{Discontinuation sequence for \url{drivers.xml}}

\begin{description}
\pause
\item[Jan 2012] Removal of XML-based load mechanism in CTRV, therefore
	in all CTR drivers
\pause
\item[Feb 2012] Full move to \url{transfer.ref}-based mechanism
	for Skel-based, Linux drivers (CVORG, MTT, XMEM, VD80)
\pause
\item[Apr 2012] Removal of XML-based load mechanism for Skel-based,
	LynxOS drivers (the same)
\end{description}

\pause
From that point on, all driver installations will be based on
information extracted via \url{transfer.ref}
\end{frame}

\section{Loading PCI drivers}

\begin{frame}{Problems with PCI configuration in the CCDB}

\begin{itemize}
\pause
\item PCI devices are auto-discovered and assigned topological
coordinates (BDF) at boot time
\pause
\item The pysical position of a board in a slot is very weakly related
to its BDF coordinates (the only ones known to the drivers)
\pause
\item crate configuration in the CCDB associates physical position
in crates with board identity (HWTYPE and LUN) and, therefore, its
hardware parameters (base addresses, interrupt vectors, etc.)
\end{itemize}
\end{frame}

\begin{frame}{VME versus PCI}
\begin{description}


\pause
\item[In VME]
\begin{equation*}
	\text{HWTYPE LUN} \leftrightarrow
	\text{PHYS-SLOT} \leftrightarrow
	\text{BASE-ADDR}
\end{equation*}
\pause
Everything is good and under our control

\pause
\item[In PCI]
\begin{equation*}
	\text{HWTYPE LUN} \leftrightarrow
	\text{PHYS-SLOT} \leftrightarrow
	\text{BDF}
\end{equation*}
\pause
The last dependency is \emph{implicit} (no control)

\pause
\item[To know more] Look into \\
\url{http://wikis.cern.ch/download/attachments/38076646/pci-ccdb.pdf}
\end{description}
\end{frame}

\begin{frame}{Proposals}
\begin{itemize}
\pause
\item the problem should be tackled soon by appropriate description
	of PCI hardware in the CCDB
\pause
\item a task group to tackle the problem with members from
	FE, DM, HT should be created to discuss/implement a solution
\pause
\item a proposal is already sketched in the document referred above
\end{itemize}
\end{frame}

\section{Other aspects}

\begin{frame}{Other aspects}
\begin{itemize}
\pause
\item drivers missing/needed for the renovation process (VTSM, others?)
\pause
\item something else?
\end{itemize}
\end{frame}

\end{document}

